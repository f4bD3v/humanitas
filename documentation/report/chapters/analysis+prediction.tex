\section*{Price Transmission Analysis}

\subsection*{Interpretation}
automate interpretation to a certain extent by learning about circumstances through online data.

\subsection*{Time Series Analysis}
Time series data has a natural temporal relation between different data points.
It is important in the analysis to extract significant temporal statistics out
of data. We will focus on analyze stationarity, autocorrelation, trend, volatility
change, and seasonality of our price datasets in R.

Stationarity of a series guarantees that the mean and variance of the data do not
change over time. This is crucial for a meaningful analysis, since if the data is
not stationary, we can not be sure that anything we derive from the present will
be consistent in the future. We can transform our data into a stationary one by
taking k-th difference to remove the underlying trend, and then apply standard
test procedures such as KPSS test [1] to see if the differenced series is stationary.

Autocorrelation is another important trait in time series data. It suggests the
degree of correlation between different time periods. By plotting correlograms
(autocorrelation plots) of our data, we will be able to identify if the fluctuation
of prices may be due to white noise or other hidden structures.

Seasonality is reasonably expected in our agricultural related time series. Several
methods might help us to detect seasonality, such as common run charts, seasonal
subseries plots, periodograms, and the correolograms we mentioned before.

(trend and volatility change is straightforward and can be concluded once we have the datasets)

[1] Kwiatkowski, D.; Phillips, P. C. B.; Schmidt, P.; Shin, Y. (1992). "Testing the null hypothesis of stationarity against the alternative of a unit root". Journal of Econometrics 54 (1–3): 159–178.


\section*{Prediction Models}
DATA: regularly sampled time series: wholesale daily, retail daily and weekly

\subsection*{Time Series Forecasting}

\subsubsection*{ARMA Model}
The classical Time series forecasting approach is to use the ARMA (Auto-Regressive
Moving Average) model to predict the target variable as a linear function which
consists of the auto-regressive part (lag variables) and the moving average part
(effects from recent random shocks).

The ARMA(p,q) model: (will refine math representations later)

$Phi(B) * Y_t = Theta(B) * eps_t$

The fitting of the model and the historical
data can be accomplished by maximum likelihood estimation.

\subsubsection*{Regression}
We can also apply ARMA to the linear regression model. It is formulated as such:

$Y = Beta*X + eps,   eps ~ ARMA(p,q)$

Through OLS (Ordinary Leasat Square) or GLS (General Least Square) processes,
we can obtain an optimal Beta.

\subsection*{Multilayer Perceptrons}
taken from M. Seegers course on Pattern Recognition and ML

\subsection*{Recurrent Neural Networks (RNN)}
source: scholarpedia
A recurrent neural network (RNN) is a neural network in with feedback connections, enabling signals to be fed back from a layer $l$ in the network to a previous layer. %The key feature is that the weight matrix for each layer $l$ in the network contains input weights from \emph{all} other neurons in the network and not just the neurons from the previous layer. 

\subsubsection*{Simple Recurrent Networks}
The simplest form of an RNN consists of an input, an output and one hidden layer as depicted in fig.[]. 

\includegraphics[width=.7\textwidth]{./img/simple_rnn.png}
[source: wikipedia]

\subsubsection*{General description of a discrete time RNN}
A discrete time RNN is a graph with $K$ input units $\vec{u}$, $N$ internal network units $\vec{x}$ and $L$ output units $\vec{y}$. The activation (per layer) vectors at point n in time are denoted by $\vec{u}(n) = (u_1(n),...,u_n(n))$, $\vec{x}(n) = (x_1(n),...,x_n(n))$, $\vec{y}(n) = (y_1(n),...,y_n(n))$. Edges between the units in these sets are represented by weights $\omega_{ij}\neq0$ which are gathered in adjacency matrices. There are four types of matrices:\par
\begin{itemize}
	\item $\vec{W}^{in}_{N\times K}$ contains inputs weigths for an internal unit in each row respectively 
	\item $\vec{W}_{N\times N}$ contains the internal weights. This matrix is usually sparse with densities $5\%-20\%$
	\item $\vec{W}^{out}_{L\times (K+N+L)}$ contains the weights for edges, which can stem from the input, the internal units and the outputs themselves, leading to the output units.
	\item $\vec{W}^{back}_{N\times L}$ contain weights for the edges that project back from the output units to the $N$ internal units
\end{itemize}

In a \emph{fully recurrent network} every unit receives input from all other units neurons and therefore input units can have direct impact on output units. Output units can further be interconnected.\par

\paragraph{Evaluation}
The calculation of the new state of the internal neurons in time-step $n+1$ is called evalution. 
\[
\vec{x}(n+1)=\vec{f}(\vec{W}^{in}\vec{u}(n+1)+\vec{W}\vec{x}(n)+\vec{W}^{back}\vec{y}(n))
\]
where $f=(f_1,...,f_N)$
\paragraph{Exploitation}
The output activations are then computed from the internal state of the network in the exploitation step.
\[
\vec{y}(n+1)=f^{out}(\vec{W}^{out}(\vec{u}(n+1),\vec{x}(n+1),\vec{y}(n)))
\]
where $f^{out}=(f^{out}_1,...,f^{out}_L)$ are the output activation functions and the matrix of output weights is multiplied by the concatenation of input, internal and previous output activation vectors.\par

%A \emph{Hopfield network} is an RNN all connections of which are symmetric and which requires stationary inputs. 
RNNs can in theory approximate any dynamical system with chosen precision, however training them is very difficult in practice. 

\subsection*{Echo State Networks}
Echo State Networks (ESN) are a type of discrete time RNNs for which training is straightforward with linear regression methods.  The temporal inputs to the network are transformed to a high-dimensional \emph{echo state}, described by the neurons of a sparsely connected \emph{random} hidden layer which is also called a reservoir. The output weights are the only weights in the network that can change and are trained in a way to match the desired output. ESNs and the related liquid state machines (LSMs) form the field of \emph{reservoir computing}.

\subsubsection{Echo State Property}
The intuitive meaning of the \emph{echo state property} (ESP) is that the internal state is \textbf{uniquely} determined by the history of the input signal and the teacher forced output, given that the network has been running long enough. Teacher forcing essentially means that the output $\vec{y}(n-1)$ is forced to be equal to the next time series value $\vec{u}(n)$ and thus to the next input.
\begin{frm-def}
For every left infinite sequence $(\vec{u}(n),\vec{y}(n-1)),n=\dots,-2,-1,0$ and all state sequences $\vec{x}(n),\vec{x'}(n)$ which are generated according to
\begin{align*}	
	\vec{x}(n+1)=\vec{f}(\vec{W}^{in}\vec{u}(n+1)+\vec{W}\vec{x}(n)+\vec{W}^{back}\vec{y}(n))\\
	\vec{x'}(n+1)=\vec{f}(\vec{W}^{in}\vec{u}(n+1)+\vec{W}\vec{x'}(n)+\vec{W}^{back}\vec{y}(n))
\end{align*}
it holds true that $\vec{x}(n)=\vec{x'}(n)$ for all $n \leq 0$.
\end{frm-def}
The echo state propery is ensured through the matrix of internal weights $W$

\begin{frm-thm}
Define $\sigma_{max}$ as largest singular value of $\vec{W}$, $\lambda_{max}$ as largest absolute eigenvalue of $\vec{W}$.
\begin{enumerate}
\item If $\sigma_{max} < 1$ then the ESP holds for the network
\item If $\|\lambda_{max}\| > 1$ then the network has no echo states for any input/output interval which contains the zero input/output tuple (0,0)
\end{enumerate}
\end{frm-thm}
In practice is suffices to make sure the negation of the second point holds. 

\subsubsection*{Training the ESN}
"The state of the ESN is therefore a function of the finite history of the inputs presented to the network. Now, in order to predict the output from the states of the oscillators the only thing that has to be learned is how to couple the outputs to the oscillators, i.e. the hidden to output connections:"
http://stackoverflow.com/questions/21940860/echo-state-network-learning-mackey-glass-function-but-how

Hyperparameters: dimensionality of W, spectral radius $\alpha$

\paragraph*{Initial state determination}
The network is run for a first set of inputs and the results are then discarded. If the spectral radius is close to unity, implying slow forgetting of the starting state, the initial set has to be a substantial part of the training dataset.

"Likewise, when the trained network is used to predict a sequence, a long initial run of the sequence must be fed into the network before the actual prediction can start. This is a nuisance because such long initial transients are in principle unneccessary"

"By contrast, a recurrent neural network such as our echo state network, but also such as the networks used in [5] need long initial runs to “tune in” to the to-be-predicted sequence. [5] tackle this problem by training a second, auxiliary “initiator” network."
EchoStatesTechRep.pdf, p.32

\paragraph*{Teacher forcing}

\paragraph*{Feedback}

\paragraph*{Batch learning with Ridge Regression}

\paragraph*{Leaky integrator neurons}
\emph{Taken from Echo State Tech Rep. page 26/27}\\
% exploitation: \vec{f}(\vec{W}^{in}\vec{u}(n+1)+\vec{W}\vec{x}(n)+\vec{W}^{back}\vec{y}(n))
In order for the Echo State Network to be able to learn slowly and continuously changing dynamics and thereby to capture longterm phenoma in the price sequences we feed in, we need a way to introduce continuous dynamics. This is done via approximation of the differential equation of a continuous-time leaky integrator network
\begin{equation}
    \frac{d\vec{x}}{d\vec{t}} = C (-\alpha\vec{x} + \vec{f}(\vec{W}^{in}\vec{u}(n+1)+\vec{W}\vec{x}(n)+\vec{W}^{back}\vec{y}(n)))
\end{equation}
where C is a time constant and $\alpha$ the leaking decay rate. For the approximation we introduce a stepsize $\delta$:
\[
    \vec{x}(n+1)=(1-\delta C \alpha) + \delta C (\vec{f}(\vec{W}^{in}\vec{u}(n+1)+\vec{W}\vec{x}(n)+\vec{W}^{back}\vec{y}(n)))
\]

\begin{frm-thm}
Let a network be updated according to
\end{frm-thm}

\paragraph*{Online learning}
Very simple: Least Mean Squares, but bad performance
update weights in each iteration
...
\begin{frm-algo}
\begin{enumerate}
	\item

	\item

\end{enumerate}
\end{frm-algo}

Recursive Least Squares instable during training
Therefore use (inverse) QR RLS, for motivation visit this link:
http://zone.ni.com/reference/en-XX/help/372357A-01/lvaftconcepts/aft_rls_algorithms/

\paragraph*{Parameter selection with Maximum Entropy Bootstrap (Meboot)}
In order to find the best parameters for generalization during training of the neural network models with we create replicate time series of a selected price sequence dataset. The method we employ to this end is called 'Maximum Entropy Bootstrap' (meboot) and was introduced by H.D. Vinod in 2006. [reference]. The reason for the use of this specific method is that, due to temporal dependence, time series cannot simply be randomly sampled into a new dataset. The meboot algorithm allows for construction of random replicates of the given time series showing the same statistical properties.

\subsection*{RNN with backpropagation decorrelation algorithm}

\subsection*{Data Mining}

In this section we give a brief introduction to association rule mining and detail the implementation and evaluation of the system. In order to run the code you will need to install the libraries Scipy and Orange 

\subsubsection{A brief introduction to Association rule mining}

The objective of association  mining is the elicitation of useful rules from which new knowledge can be derived. Association mining applications have been applied to many different domains including market basket analysis, risk analysis in commercial environments, clinical medicine and crime prevention. They are all areas in which the relationship between objects can provide useful knowledge. 

Itemsets are identified by the use of two metrics support and confidence. 

Support is a measure of the statistical significance of the rule. Rules with a very low support are more likely to occur by chance. With respect to the market basket analysis items that are seldom bought together by customers are not profitable to promote together. For this reason support is often used as a filter to eliminate uninteresting rules. 

Confidence on the other hand is is a measure on how reliable the inference made by a rule is. For a given rule $A \implies B$, the higher the confidence, the more likely it is for the item set B to be present in the transactions that contain A. IN a sense confidence provides an estimate of the conditional probability for B given A. 

It is worth noting that the inference made by an association rule does not necessarily imply causality. Instead the implication indicates a strong concurrence relationship between items in the antecedent and consequent of the rule. 

So how do we go about implementing this in an algorithm? A rather naive approach would be to check if each itemsset satisfies minimum support. However this is rather inefficient and unnecessary. We can make use of of the observation that every subset of a frequent item set also has to be a frequent item set. This works the otherway as well. If a set is not frequent then its superset can not be frequent either. This observation allows us to prevent unnecessary computation. The Apriori Algorithm uses this downward closure to identify frequent item sets. Candidates that do not satisfy minimal support are pruned, which automatically reduces the algorithm's search space. Once frequent item sets have been generated the algorithm enters the second face, namely generating derivations for which the metric minimal confidence is used. 

\subsubsection{Implementation}

The implementation is guided through three stage namely data crunching, classification and  mining where the later is a straight forward implementation of the mining library orange. \\



In order to run the Apriori Algorithm we needed to do some preprocessing since the data available was in a uncompatible format. The data readily available to us was in the form 
of Date Country City Category Commodity 1, Date Country City Category Commodity 2. In the statistics community this kind of format is referred to as the "long format". In order to run Apriori we need to convert the table to a "wide format" meaning that all commodities need to be related to one index. You can imagine a matrix where the y axis is described by city + date and the x axis is labeled with all the available commodities. To do the conversion we used Stata, a statistical software which is mostly used in the field of social science. Before reading the data into state we filtered the document for special characters ("", /) and replaced all unavailable price informations with "NA". For this purpose we used the filter.py script. Once we read in the data into state we removed duplicates and started the conversion. To reproduce the correct table format you fan follow our implementation in the stata script.  The data is now in the correct table format what remains to be done is slicing the table according to a city and sorting the data in order of decreasing time stamps. These steps are performed in the dataFrame.py. We now continue to process the files by classifying continues variables into categories. We therefor filtered the maximum and minimum increase/decrease in price to establish a range of values. Depending on whether a price increase fall into the first, second or third third of the price range we classified it as small, medium or big increase/decrease respectively. The categorial data can now be processed by the assoc.py file. The library orange provides an implementation of the a priori gen algorithm. We simply set the min support value and write the 10 rules with the highest support count to a file. 

\subsubsection{Results}

What we were hoping to find were seasonal related price changes such as those we experience when shopping for fruits and vengtables at our local grocery stores. Rules such as $Tomatoes = big increase \implies potatoes = small decrease$ could serve 


Our initial granularity was set to weekly data. From our meta analysis we observed that prices quite frequently stay unchanged over the period of several weeks. This resulted in an overwhelming amount of rules in the form of commodity $A = unchanged \implies B= unchanged$. This made it really hard to filter the set for insightful rules. We therefor decided to compare price changes over the period of 12 weeks. Although the majority of the rules still remained in the above form we managed to extract some relations related to price changes. 

We conclude that the strongest correlations exist between commodities with unchanged prices. This observation underlines the nature of food commodities. Agricultural commodities are known to be less volatile then for example energy commodities. We further noticed that depending on the region we would have a totally different rule count. Capital cities tend to have a bigger rule base meaning that there exist a higher correlation between products. Two extreme example are Dassau, which resulted in no patterns at all and Shillong which produced over 260000 rules. In addition our experiments showed the finer the granularity of the data the higher the support and correlation between products. This is only natural given that prices can stay stable over the period of weeks or even months. 

\subsubsection{Trading advice }

As concluded above our algorithm found many associations between products. In order to interpret most of the results specialized domain knowledge in trading with commodities is  necessary. Some more obvious rules we managed to interpret were the following:

$BreadLocal=big increase MilkCowBuffalo=unchanged increase \implies MaidaNA=unchanged increase$. If prices of bread inflate it is save to trade Maida, as its price will most certainly stay stable. $BiscuitGlucose=unchanged increase \implies BesanNA=small increase$, similarly means that stable pries in biscuits will imply a small increase in Besan. 



\begin{comment}
\subsubsection*{Finding the right topology for the specific prediction problem}

\subsubsection*{Converge}
conceive network that converges fast to speed up training\par
"Known supervised training techniques for RNNS comprise Back Propagation Through Time (BPTT), Real Time Recurrent Learn- ing (RTRL) or Extended Kalman Filtering (EKF) all of which have some major drawbacks."
{\em Application of BPTT to RNNs requires stacking identical copies of the network thus unfolding the cyclic paths in the synaptic connections. Unlike back-propagation used in feed-forward nets, BPTT is not guaranteed to con- verge to a local error minimum, computational cost is O(TN2) per time step where N is the number of nodes, T the number of epochs. In contrast RTRL needs O((N + L)4) (L denotes number of output units), which makes this algorithm only applicable for small nets. The algorithm complexity of EKF is O(LN2). EKF is mathematically very elaborate and only a few experts have trained predefined dynamical system behaviors successfully}\par


\subsubsection*{Avoiding overfitting}

\cite{jaeger_echo_state_RNN}
The third section explains how echo state networks can be trained in a
supervised way. The natural approach here is to adapt only the weights
of network-to-output connections. Essentially, this trains readout functions
which transform the echo state into the desired output signal. Technically,
this amounts to a linear regression task.
\end{comment}


