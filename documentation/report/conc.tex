\section*{Conclusion}
Twitter statistics however very much favor Indonesia with 11.7\% of the population being users while in India the percentage is only 1.3\%. Were we to continue with the project it would make sense to apply the methods we conceived to the newly available \href{http://www.kemendag.go.id/en/economic-profile/prices/national-price-table}{national price table} published by the Trade Ministry of Indonesia. 

An interesting additional feature would be to derive food supply indicators from social media and to figure out if it is possible to detect underlying patterns in food supply in a country that are actually caused by policy decisions or inaction. This could aid the respective government in allocating resources in their substantial intervention in the food market.

\section*{Closing remarks}